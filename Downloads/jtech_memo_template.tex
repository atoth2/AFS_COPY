\documentclass[12pt, letterpaper]{article}
%
% File: jtech_memo_template.tex
%
% Written in MS Word by Dr. David Go, Fall 2011 for AME20213
% Translated to LaTex by John Ott
%
% Contact John Ott for any packages you don't have install
% on your latex system.
%
% when printing, "print actual size", "not fit to page"
%
%%%%%%%%% EXACT 1in MARGINS %%%%%%%
%
\setlength\voffset{0pt}%
\setlength\headsep{25pt}% 25pt
\setlength\headheight{12pt}% 12pt
\setlength\topmargin{0pt}%
\addtolength\topmargin{-\headheight}%
\addtolength\topmargin{-\headsep}%

\setlength\hoffset{0pt}
\setlength\marginparwidth{0pt}
\setlength\oddsidemargin{0pt}

\setlength\textwidth{\paperwidth}
\addtolength\textwidth{-2in}
\setlength\textheight{\paperheight}
\addtolength\textheight{-2in}

\setlength{\parindent}{0pt}
\setlength{\textfloatsep}{10pt plus 2pt minus 4pt}

% remove double spacing between bibitem entries
\let\OLDthebibliography\thebibliography
\renewcommand\thebibliography[1]{
  \OLDthebibliography{#1}
  \setlength{\parskip}{0pt}
  \setlength{\itemsep}{0pt plus 0.3ex}
}

\usepackage{graphicx}
\usepackage{subfig}
\usepackage[justification=centering]{caption}
\usepackage{setspace}% for double spacing
\usepackage{mathptmx}% for adobe roman font
\usepackage{amsmath}
\usepackage{tabularx}
\usepackage{booktabs}% to deal with hline spacing issues in tables

\newcommand{\tab}{\hspace*{2em}}
\newcommand{\HRule}{\rule{\linewidth}{0.5mm}}
\newcommand{\z}[1]{\text{#1}} % use for times roman font in equations

%%%%%%%%%%%%%%%%%%%%%%%%%%%%%%%%%%%%%%%
\begin{document}
%%%%%%%%%%%%%%%%%%%%%%%%%%%%%%%%%%%%%%%


\makeatletter
\renewcommand\paragraph{% start new line after key
    \@startsection{paragraph}{4}{\z@}%
    {-3.25ex\@plus -1ex \@minus -.2ex}%
    {0.01em}%
    {\normalfont\normalsize\bfseries}}
\makeatother

\noindent {\bf AME 20213: Measurements and Data Analysis} \\
{\bf Technical Memo} \\
\\{\bf Date Submitted: }
\\{\bf Dates Performed: } \textit{day/time} \\
\\{\bf To: }
\\{\bf From: } \\
\\{\bf Subject: } Lab Exercise 1 - Measurement Systems/Calibration

\doublespacing
\paragraph{Summary:} In this work, two types of studies were conducted. The
first was the calibration of a large-scale strain gage and the second was the
use of cantilever beam to measure the mass of a penny and an unknown material
(a cylinder), which was then identified by its density. For the calibration
study, the large-scale strain gage showed a linear relationship with a high
coefficient of determination ($r^2$ = 0.999) and a small relative standard
error that was far less than 1\% at 95\% confidence. It was also determined
that the cantilever beam apparatus in full bridge mode was more sensitive with
narrower confidence intervals than in the quarter bridge configuration. The
measured mass of the penny for the two configurations was 4.08 g and 4.10 g.
respectively, both greater than the published value of 2.500 g by more than
70\% This was attributed to dirt and other contaminants on the pennies skewing
the data. The mass for the cylinder was measured to be 305.8 g (full bridge)
and the volume was 35.941 $\z{\z{cm}}^3$, such that the density was 8.501
$\z{g/cm}^3$ in the full bridge configuration, and 288.7 g and 8.058
$\z{g/cm}^3$ in the quarter bridge configuration. In comparison with published
data of common materials, the unknown material was identified to most likely be
brass or bronze.

\paragraph{Findings:}
In this work, two studies were conducted: 1) a `macro'
strain gage was built and calibrated, and 2) strain gages in two different
Wheatstone bridge configurations (full and quarter) on a cantilever beam were
used to determine the mass of both a single penny and the density of an
unknown object (a cylinder). For both studies, calibration was required and
linear regression analysis was implemented utilizing the least squares
approach, and quantitative conclusions were drawn based on the data. \\

\textit{Study 1:} The large-scale strain gage was constructed from a
stretching apparatus, dial, gauge, stainless steel wire, and a
multimeter. The wire was stretched known amounts and the resistance was
measured across the entire length of the wire. The resulting measured
resistance was related to the strain, defined as $\epsilon = \Delta L/L$
where $L$ is the initial wire length and $\Delta L$ is the known change
in wire length, and a calibration curve was fit to the resulting
resistance data using the unweighted least squares method. Recalling
from \cite{ame} that for linearly related  properties the
calibration curve will take the following form

\begin{equation}
R = a_0 + a_1\epsilon,
\end{equation}

the unknown coefficients can be determined by

\begin{eqnarray}
   a _1 &=& \frac{\overline{\epsilon R}
     -(\overline{\epsilon})(\overline{R})}{\overline{(\epsilon^2)}
     -(\overline{\epsilon})^2} \\
     a_0 &=& \overline{R} - a_1 \overline{\epsilon},
\end{eqnarray}

where the overbar indicates a mean quantity. Details of this calculation are
included in Appendix A. The resulting data, calibration curve, and confidence
(precision) interval are plotted in Fig. 1, where

\begin{equation}
R = 211.96 + 515.64 \epsilon  \qquad (\Omega),
\end{equation}

\begin{figure}[h!]
\begin{center}
\includegraphics[width=3in]{pic1.eps}
\caption{Calibration curve relating the resistance $R$ \\ to the known strain
              $\epsilon$ for the 'macro' strain gage.}
\label{pic1.fig}
\end{center}
\end{figure}

Several remarks can be made about the sensitivity of the calibration curve,
quality of the curve fit, and the range to which it applies. Table 1 contains
useful statistical parameters pertaining to the data. The slope of the
regression line $K$ is referred to as the sensitivity of the calibration curve
and takes of a value of $K$ = 515.65 $\Omega$. The slope is positive but of
$\mathcal{O}(10^0)$ suggesting relatively low sensitivity. The standard error
$t_{10,95\%} S_{yx}$ was $5.7928 \times 10^{-2}$ at 95\%  confidence where the
Student's $t$ value was $t_{10,95\%} = 2.228$ \cite{dunn}. Given that the measure
resistance is $\sim$ 200 $\Omega$, the relative standard error is less than
$2.6811 \times 10^{-2}$  \%, suggesting that the curve fit was good.  Similarly, the
coefficient of determination $r^2$ is asymptotically close to 1 implying a
near perfect correlation, giving high confidence that the relationship is in
fact linear.  Because the strain gage was only calibrated across a strain of
$\epsilon = 0.3 - 0.4$, this is the suitable range of use for this system. Due
to the relative sparseness of data below this range and the prospect of
possible mechanical failure above it, this system is not recommended for use
outside this range without both acquiring more data or mechanically fortifying
the system. Details of the calculations used to obtain the values in Table 1
can be found in Appendix A.

\begin{table}[h!]
\centering
   \caption{Curve fit parameters for the large-scale strain gage. \\
      The degrees of freedom was $\nu =7$ and $R_{ci}$ represents calculated \\
   values from Eq. (3) while $R_i$ represent raw resistance data.}
   \begin{tabular}{ c c c c } \toprule
      Parameter & Name & Basic Equation & Value \\ \midrule
      $K$ & sensitivity & $\dfrac{dR}{d\epsilon}$ &  515.65 \\ \midrule
      $S_{yx}$ & standard error & $\sqrt{\dfrac{\sum_{i=1}^N (R_{ci} - R_i)^2}{\nu}}$ &
      $5.7928 \times 10^{-2} \quad {\Omega}$  \\  \midrule
      $r^2$ & coefficient of determination & $1 - \dfrac{\sum_{i=1}^N (R_{ci}
   - R_i)^2}{\sum_{i=1}^N(R_i - \overline{R})^2}$ & 0.99964 \\ \bottomrule
\end{tabular}
\label{table1}
\end{table}

\mbox{} \\ % add blank line
\textit{Study 2:} For the second study, a cantilever beam was outfitted
with four strain gages. In a Wheatstone bridge configuration. All four
strain gages could be utilized simultaneously in a full bridge
configuration or three of them could be replaced by fixed resistors
($R_{fix}  = 120 \Omega$) in a quarter bridge configuration. Weight
could be applied to the end of the cantilever beam and the resulting
deflection sensed by the strain gage measurement system, thus it could
be used as a scale. A calibration curve was generated for each
configuration of the cantilever beam apparatus relating the measured
mass and output voltage from the measurement system.  For a known mass
$m$, voltage $V_{out}$out was  measured  over  the  Wheatstone  bridge
circuit  and  the corresponding calibration curves along with precision
intervals are shown in Fig. 2. The calibration curve for the full bridge
configuration yielded a smaller precision interval indicating that the
full bridge produced more consistently linear data and a higher quality
curve fit.  It is worth comparing the slope of the calibration curves
for each configuration.  For the full bridge $K_{FB} = 0.012613$ while
for the quarter bridge $K_{QB} = 0.012558$, which are less than 1\%
different indicating that the full bridge configuration was negligibly
more sensitive.

\begin{figure}[!ht]
   \begin{center}
   \subfloat[ \label{subfig-1:dummy}]{%
      \includegraphics[width=0.45\textwidth]{pic2a.eps}
    }
    \hfill
    \subfloat[ \label{subfig-2:dummy}]{%
      \includegraphics[width=0.45\textwidth]{pic2b.eps}
    }
   \caption{Calibration curve relating the output voltage Vout to the known mass
       $m$ for (a) the full bridge configuration and (b) the quarter bridge configuration.}
    \label{fig:dummy}
 \end{center}
\end{figure}

\mbox{} \\
This apparatus was used to measure the mass of a single penny as well as
several discrete quantities of pennies. The mass of a single penny is 2.500 g
according to the United States mint \cite{us}, and the measured results were
compared graphically to the predicted mass $m = 2.5n$ (g) in Figure 3, where
$n$ is the number of pennies. A simple qualitative assessment of the data
shows a decline in accuracy as the measured number of pennies increases as
both configurations over predict the theoretical mass. This is can likely be
attributed at least in part to the aggregate amount of dirt and other
contaminants increasing as the number of pennies increases. For the two
measurement configurations there is a slight discrepancy in the measured value
of single penny; however, it would seem that as the mass increases the two
techniques yield results that steadily approach each other. Table 2 lists the
masses measured for the two configurations and the percent difference relative
to the theoretical value. For this measurement the quarter bridge method
yielded data closer to the published value, but this is possibly due to a less
sensitive calibration curve (as shown in Fig. 2b) as opposed to an inherently
more accurate system.


\begin{figure}[h!]
\begin{center}
\includegraphics[width=3.25in]{pic3.eps}
\caption{Measured mass $m$ as a function of the number of pennies $n$
   for the full and quarter bridge configuration along with the
predicted value from [3].}
\label{pic3.fig}
\end{center}
\end{figure}

\begin{table}
\centering
\caption{A comparison of the measured and predicted values \\  for the two measurement configurations.}
\begin{tabular}{c c c c c c} \toprule
number      & predicted mass & measured mass & \% difference & measured mass  &\% difference \\
of pennies  & (g)                   &  full bridge (g)   &                       & quarter bridge (g)  &   \\ \midrule
1 & 2.500 & 6.6252 & 265.0 & 3.751 & 150.0 \\ \midrule
10 & 25.00 & 29.62 & 11.85 & 26.84 & 10.74 \\ \midrule
25 & 62.50 & 65.30 & 4.180 & 64.27 & 4.110 \\ \midrule
50 & 125.0 & 131.1 & 2.100 & 129.6 & 2.070 \\ \bottomrule
\end{tabular}
\label{table 2}

\end{table}

The cantilever beam apparatus was also used to measure the mass of an object
of unknown material, in this case a cylinder. The volume $V$ of the material
was determined based on measurements of the length $L$ and diameter $d$ using
calipers and the relationship $V = (1/4)\pi d^2 L (\z{cm}^3)$. The ensuing
density $\rho$ was than calculated as

\begin{equation}
   \rho = m/V \qquad (\z{g/cm}^3).
\end{equation}

Table 3 outlines the measured values using the full and quarter bridge
configurations. Similar to the penny measurements, the quarter bridge
configuration measured a mass lower than the full bridge, resulting in an
approximately 5\% difference in calculated density.   Based on these
measured values, there are a wide range of candidate materials based on a
density between 8.0 - 8.5 $\z{g/cm}^3$  for the cylinder according to
\cite{thermtest}, including brass alloys, bronze alloys, cadmium, cobalt
alloys, and stainless steels.
Based on the color and likely cost, many of these can be disregarded and a
reasonable judgment is that the cylinder is made from a brass or bronze alloy,
It is important to note that this is just speculation, and without knowing
more material properties a definitive material classification is not possible.
Again, detailed calculations are included in Appendix A.

\begin{table}
\begin{center}
   \caption{The calculated density for the cylinder of unknown material.}
\begin{tabular}{ccccc} \toprule
   volume $(\z{cm}^3)$ & $m_{\z{FB}}$ (g) & $\rho_{\z{FB}} (\z{g/cm}^3)$
   & $m_{QB}$ (\z{g}) & $\rho_{\z{QB}} (\z{g/cm}^3)$ \\ \midrule
35.940 & 305.8 & 8.501 & 288.7 & 8.058 \\ \bottomrule
\end{tabular}
\label{table3}
\end{center}
\end{table}

\paragraph{Conclusions:}
A large-scale strain gage was constructed and
calibrated. While the error associated with this system is minimal, its range
of operation is limited. Strain gages in Wheatstone bridge configuration fixed
to a cantilever beam created a mass measurement system that was calibrated and
utilized to measure the mass of a penny and an unidentified material. The
system generally over predicted the theoretical mass of the penny. Finally
based on mass and volume measurements, the density of a cylinder of unknown
material was measured and identified to likely be brass or bronze.

% The following can be deleted and manually type in your references

\paragraph{References:}
\begingroup
   \def\section*#1{}% remove References from thebibliography environment
\raggedright
\singlespacing
\begin{thebibliography}{}
   \bibitem{ame} AME20213, Fall 2011, "Measurement  Systems/Calibration",
       Lab Exercise \#1 Week A, University of Notre Dame, Notre Dame, IN.
   \bibitem{dunn}  Dunn, P. F., 2009, Measurement and Data Analysis,
       University of Notre Dame, Notre Dame, IN, Chap. 6.9.
   \bibitem{us} The United States Mint, 2011, "Coin Specifications",
        http://www.usmint.gov/about\_the\_mint/?action=coin\_specifications.
    \bibitem{thermtest} ThermTest Inc., 2011, "Material Property Search Results",
        http://www.thermtest.com/material-property-search/.
\end{thebibliography}
\endgroup

\end{document}

